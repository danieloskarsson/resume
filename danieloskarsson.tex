% !TEX TS-program = xelatex
% !TEX encoding = UTF-8 Unicode
% -*- coding: UTF-8; -*-
% vim: set fenc=utf-8

%%%%%%%%%%%%%%%%%%%%%%%%%%%%%%%%%%%%%%%%%%%%%%%%%%%%%%%%%%%%%%%%%
%% CV.tex
%% <https://github.com/zachscrivena/simple-resume-cv>
%% This is free and unencumbered software released into the
%% public domain; see <http://unlicense.org> for details.
%%%%%%%%%%%%%%%%%%%%%%%%%%%%%%%%%%%%%%%%%%%%%%%%%%%%%%%%%%%%%%%%%

% See "README.md" for instructions on compiling this document.

\documentclass[letterpaper,MMMyyyy,nonstopmode]{resume}
% Class options:
% a4paper, letterpaper, nonstopmode, draftmode
% MMMyyyy, ddMMMyyyy, MMMMyyyy, ddMMMMyyyy, yyyyMMdd, yyyyMM, yyyy

%%%%%%%%%%%%%%%%%%%%%%%%%%%%%%%%%%%%%%%%%%%%%%%%%%%%%%%%%%%%%%%%%
%% PREAMBLE.
%%%%%%%%%%%%%%%%%%%%%%%%%%%%%%%%%%%%%%%%%%%%%%%%%%%%%%%%%%%%%%%%%

\usepackage{verbatim}
\usepackage{xstring}
\usepackage{catchfile}
\CatchFileDef{\HEAD}{.git/refs/heads/svenska}{}
\newcommand{\gitrevision}{%
  \StrLeft{\HEAD}{7}%
}

% CV Info (to be customized).
\newcommand{\CVAuthor}{Daniel Oskarsson}
\newcommand{\CVTitle}{Daniel Oskarsson's resume}
\newcommand{\CVNote}{CV:t skapades {\today} från {\gitrevision} med {\XeLaTeX}}
\newcommand{\CVWebpage}{https://github.com/danieloskarsson}

% PDF settings and properties.
\hypersetup{
pdftitle={\CVTitle},
pdfauthor={\CVAuthor},
pdfsubject={\CVWebpage},
pdfcreator={XeLaTeX},
pdfproducer={},
pdfkeywords={},
unicode=true,
bookmarks=true,
bookmarksopen=true,
pdfstartview=FitH,
pdfpagelayout=OneColumn,
pdfpagemode=UseOutlines,
hidelinks,
breaklinks}

% Shorthand.
\newcommand{\Code}[1]{\mbox{\textbf{#1}}}
\newcommand{\CodeCommand}[1]{\mbox{\textbf{\textbackslash{#1}}}}

%%%%%%%%%%%%%%%%%%%%%%%%%%%%%%%%%%%%%%%%%%%%%%%%%%%%%%%%%%%%%%%%%
%% ACTUAL DOCUMENT.
%%%%%%%%%%%%%%%%%%%%%%%%%%%%%%%%%%%%%%%%%%%%%%%%%%%%%%%%%%%%%%%%%

\begin{document}

%%%%%%%%%%%%%%%
% TITLE BLOCK %
%%%%%%%%%%%%%%%

\Title{\CVAuthor}

\begin{SubTitle}
\href{https://www.google.com/maps/place/VEKABACKEN,+521+94+Falk%C3%B6ping/@58.3358392,13.7009896,12z/data=!4m5!3m4!1s0x465ae469d63424bf:0xb367c493ed2f945c!8m2!3d58.2692871!4d13.5626312}
{Sätuna Vekabacken, 521 94 Falköping}
\par
\href{mailto:daniel.oskarsson@gmail.com}
{daniel.oskarsson@gmail.com}
\,\SubBulletSymbol\,
0705-904720
\\
\href{\CVWebpage}
{\url{\CVWebpage}}
\end{SubTitle}

\begin{Body}

%%%%%%%%%%%%
%% SKILLS %%
%%%%%%%%%%%%

\Section
{Kunskaper}
{Skills}
{PDF:Skills}

\Entry
Som profesionell utvecklare har jag arbetat med en mängd olika programmeringsspråk, exempelvis kotlin/java8+, swift/objective-c, html5/css3, typescript/javascript och c\#. På hobbynnivå har jag också använt mig av c/c++/golang, python, m.fl. Jag har stor kunskap om mobilutveckling för Android och iOS samt goda kunskaper i applikationsutveckling med Java (jpa/jsp/ejb) och Spring Boot. Jag ser praktiker som agil utvecklingsmetodik och att skriva tester som en självklarhet.

%%%%%%%%%%%%%%%%%%%%%%%%%%%
%% WORK EXPERIENCE %%
%%%%%%%%%%%%%%%%%%%%%%%%%%%

\Section
{Anställningar}
{Work Experience}
{PDF:OtherWorkExperience}

\Entry
\href{http://his.se}
{\textbf{Högskolan i Skövde}}, Skövde

\Gap
\BulletItem
Lärare (deltid)
\hfill
\DatestampYMD{2020}{02}{01} --
\newline
Institutionen för informationsteknologi
\begin{Detail}
\SubBulletItem
Mobil applikationsutveckling: kursutveckling, föreläsningar och handledningar
\end{Detail}

\begin{comment}
\Gap
\BulletItem
Studentmentor (deltid)
\hfill
\DatestampYMD{2018}{09}{01} --
\DatestampYMD{2019}{07}{01}
\newline
Avdelningen för utbildnings- och studentstöd
\begin{Detail}
\SubBulletItem
Särskilt efterfrågad av student som mentor
\end{Detail}

\Gap
\BulletItem
Handledare (deltid)
\hfill
\DatestampYMD{2006}{01}{01} --
\DatestampYMD{2006}{06}{01}
\begin{Detail}
\SubBulletItem
Vidareutveckling, handledning och rättning av projekt i mjukvarutveckling
\end{Detail}
\end{comment}

\BigGap
\Entry
\href{http://easce.com/}
{\textbf{EASCE}}, Skövde

\Gap
\BulletItem
Entreprenör, Utvecklare, Designer, \& Produktutvecklare
\hfill
\DatestampYMD{2020}{01}{01} --
\begin{Detail}
\SubBulletItem
Delägare i aktiebolag vars syfte är att verka som en inkubator för egna ideer. 
\end{Detail}

\BigGap
\Entry
\href{http://softwareyou.se/}
{\textbf{Software \& You}}, Skövde

\Gap
\BulletItem
Entreprenör, Utvecklare \& Designer
\hfill
\DatestampYMD{2015}{09}{17} --
\begin{Detail}
\SubBulletItem
Eget aktiebolag inom vilket jag jobbat med en mängd olika kunder i Science Park Skövde och företag i Stockholm som exempelvis Bontouch och SJ.
\end{Detail}

\BigGap
\Entry
\textbf{Saltside}, Göteborg

\Gap
\BulletItem
Engineering Manager
\hfill
\DatestampYMD{2014}{04}{01} --
\DatestampYMD{2015}{09}{01}

\begin{Detail}
\SubBulletItem
Chef för ett litet team med Android och iOS-utvecklare (av vilket många jag rekryterade) under 2014 och för ett större ''cross-functional'' team 2015. På Saltsida tog jag även fram den agila utvecklingsprocess som användes, en process centrerad kring kunskapsspridning och autonoma team baserad på Scrum och Kanban.
\end{Detail}

\BigGap
\Entry
\textbf{Duego}, Göteborg

\Gap
\BulletItem
Lead Mobile Developer
\hfill
\DatestampYMD{2012}{07}{01} --
\DatestampYMD{2014}{03}{01}

\begin{Detail}
\SubBulletItem
Ansvarig för den tekniska implementationen av företagets mobilapplikation.
\end{Detail}

\BigGap
\Entry
\textbf{Knowit}, Göteborg

\Gap
\BulletItem
Systemutvecklingskonsult
\hfill
\DatestampYMD{2010}{10}{01} --
\DatestampYMD{2012}{06}{01}

\begin{Detail}
\SubBulletItem
Som konsult på Knowit arbetade jag på plats i Göteborg hos kunder som Stendahls, AstraZeneca, Göteborgs stad och Cryptzone. Att arbeta som konsult gjorde att jag utvecklade skickligheter gällande kommunikation med intressenter med olika kunskaper. Som anställd deltog jag ofta i och utförde egna anställningsintervjuer.
\end{Detail}

\BigGap
\Entry
\textbf{Telia company}

\Gap
\BulletItem
Lösningsarkitekt, Göteborg
\hfill
\DatestampYMD{2006}{03}{01} --
\DatestampYMD{2010}{10}{01}

\begin{Detail}
\SubBulletItem
Arbetade med mängder av olika typer av projekt på telia.se och halebop.se. Några exempel är introduktionen av iPhone 3G, bundling av Spotify med bredbandsabonnemang, Mina sidor och styrning av inspelningar i Telia Digital TV via Mina sidor. Som tekniskt ansvarig för ett team med utvecklare lärde jag mig att coacha ett team med oliksinnade utvecklare att arbeta och nå gemensamma mål.
\end{Detail}

\Gap
\BulletItem
Servicedesk-tekniker, Mariestad
\hfill
\DatestampYMD{2000}{12}{01} --
\DatestampYMD{2006}{03}{01}

\begin{Detail}
\SubBulletItem
Tillhandahöll telefonsupport för hård- och mjukvara på stationära och mobila klienter till företag som Telia, Länsförsäkringar, SJ m.fl. Lärde mig mycket om kommunikation med människor med olika förkunskaper i en mängd olika situationer.
\end{Detail}


%%%%%%%%%%%%%%%%%%%%%%%%%%%%%%%
%% LICENSES & CERTIFICATIONS %%
%%%%%%%%%%%%%%%%%%%%%%%%%%%%%%%

\Section
{Certifieringar}
{Licenses \& Certifications}
{PDF:LicensesAndCertifications}

\BulletItem
Professional Scrum Product Owner (genomförs i skrivande stund)
\hfill
\DatestampYM{2020}{06}

\BulletItem
Certified ScrumMaster
\hfill
\DatestampYM{2011}{04}

\newpage

%%%%%%%%%%%%%%%%%%
%% PUBLICATIONS %%
%%%%%%%%%%%%%%%%%%

\Section
{Publikationer}
{Publications}
{PDF:Publications}

\SubSection
{Internet-Drafts}
{Internet-Drafts}
{PDF:Internet-Drafts}

% Declare a new group to limit the scope of \MaxNumberedItem to this subsection.
\begingroup
\renewcommand{\MaxNumberedItem}{[88]}

\BigGap
\BulletItem
\href{https://www.ietf.org/archive/id/draft-oskarsson-jsond-00.txt}
{\underline{D.~Oskarsson}
``JavaScript Object Notation Definition (JSOND)''\\
\textit{Internet Engineering Task Force (IETF)}, \DatestampYM{2015}{03}.}
\begin{Detail}
\Item
Definitionsspråk för kommunikation av JSON-text
\end{Detail}
  
\endgroup

%%%%%%%%%%%%%%%
%% EDUCATION %%
%%%%%%%%%%%%%%%

\Section
{Utbildningar}
{Education}
{PDF:Education}

\Entry
\href{http://www.example.com/my-university}
{\textbf{Högskolan i Skövde}},
Skövde

\Gap
\BulletItem
Kandidat:
\href{http://www.his.se/uxd}
{User Experience Design}
\hfill
\DatestampYMD{2017}{08}{01} --
%\DatestampYMD{2020}{07}{01}
\newline
Institutionen för informationsteknologi


% \Gap
% \BulletItem
% B.S.
% \href{https://his.se/utbildning/data-och-it/webbutvecklare-programmering-webug/}
% {Web development - programming}
% \hfill
% \DatestampYMD{2018}{08}{01} --
% \DatestampYMD{2019}{05}{31}

\Gap
\BulletItem
Magister:
\href{http://his.se/dvp}
{Datavetenskap}
\hfill
\DatestampYMD{2002}{08}{01} --
\DatestampYMD{2006}{06}{31}
\begin{Detail}
\SubBulletItem
Uppsats:
\href{http://urn.kb.se/resolve?urn=urn:nbn:se:his:diva-13461}
{Assessing the introduction of aspect-orientation in a real-world}
\SubBulletItem
Handledare:
Henrik Grimm, Google
\SubBulletItem
Examinator:
Mikael Johanesson
\end{Detail}
    
\BigGap
\Entry
\href{http://cityu.edu.hk}
{\textbf{City University of Hong Kong}}, Hong Kong

\Gap
\BulletItem
\href{https://cs.cityu.edu.hk/courses/exchange.html}
{Datavetenskap}
\hfill
\DatestampYMD{2019}{09}{01} --
\DatestampYMD{2019}{12}{23}
\begin{Detail}
\SubBulletItem
Studier utomlands som en del av programmet User Experience Design.\\Reste hem en månad tidigare än planerat då demonstranter intog universitetet i november. Färdigställde studierna från Sverige.\end{Detail}

\BigGap
\Entry
\href{https://www.law.lu.se/}
{\textbf{Lund Universitet}},
Lund

\Gap
\BulletItem
\href{https://www.lu.se/lubas/i-uoh-lu-JURF11}
{Introduktion till Svensk lag}
\hfill
\DatestampYMD{2019}{06}{01} --
\DatestampYMD{2019}{08}{31}

%%%%%%%%%%%%%%%%%%%%%%%%%%%
%% AWARDS & SCHOLARSHIPS %%
%%%%%%%%%%%%%%%%%%%%%%%%%%%

\Section
{Stipendium}
{Awards \& Scholarships}
{PDF:AwardsAndScholarships}

\BulletItem
Högskolans resestipendium av år 1990
\hfill
\DatestampYMD{2019}{04}{09}
\begin{Detail}
\Item
''För goda studieresultat och engagemang i bl a ledningsrådet vid institutionen för\\
informationsteknologi samt tjänstgjort som studentmentor tilldelas Daniel Oskarsson\\
högskolans resestipendium på 25 000 kronor.''
\end{Detail}

%%%%%%%%%%%%%%%%%%%%%%%%%%%%%%%%%%%%%%%%%%%%
%% PROFESSIONAL AFFILIATIONS & ACTIVITIES %%
%%%%%%%%%%%%%%%%%%%%%%%%%%%%%%%%%%%%%%%%%%%%

\Section
{Aktiviteter}
{Professional Affiliations \& Activities}
{PDF:ProfessionalAffiliationsActivities}

\Entry
\textbf{IIT ledningsråd}, Institutionen för informationsteknologi, Högskolan i Skövde

\Gap
\BulletItem
Studentrepresentant
\hfill
\DatestampYMD{2017}{11}{24} -- \DatestampYMD{2019}{06}{01}
\\
\Entry
\textbf{CocoaHeads Göteborg}, Göteborg

\Gap
\BulletItem
Arrangerade återkommande träffar för iOS-utvecklare på olika platser i Göteborg
\hfill
\DatestampYMD{2011}{03}{10} -- \DatestampYMD{2014}{06}{01}
\\
\Entry
\textbf{Datasektionen vid Högskolan i Skövde (DHISK)}, Skövde

\Gap
\BulletItem
Styrelsemedlem
\hfill
\DatestampYMD{2002}{09}{01} -- \DatestampYMD{2006}{06}{01}

%%%%%%%%%%%%%%%
%% LANGUAGES %%
%%%%%%%%%%%%%%%

\Section
{Språk}
{Languages}
{PDF:Languages}

\BulletItem
Svenska: Modersmål

\Gap
\BulletItem
Engelska: Flytande

%%%%%%%%%%%%%%%
%% INTERESTS %%
%%%%%%%%%%%%%%%

\Section
{Intressen}
{Interests}
{PDF:Interests}

\Entry
Resor, sport, entreprenörskap, utbildning, produktutveckling

%%%%%%%%%%%%%%%%
%% REFERENCES %%
%%%%%%%%%%%%%%%%

\Section
{Referenser}
{References}
{PDF:References}

\textbf{Referenser ges vid begäran}

\end{Body}

%%%%%%%%%%%
% CV NOTE %
%%%%%%%%%%%

\BigGap
\UseNoteFont%
\null\hfill%
[\textit{\CVNote}]

\end{document}
